\documentclass{article}

\usepackage[brazil]{babel}
\usepackage[T1]{fontenc}
\usepackage[a4paper, margin=1.5cm]{geometry}
\usepackage[colorlinks, urlcolor=blue, citecolor=red]{hyperref}
\usepackage[utf8]{inputenc}
\usepackage{amsmath}

\newcommand{\pslice}{\url{https://docs.python.org/3/glossary.html\#term-slice}}
\newcommand{\punpack}{\url{https://docs.python.org/3/tutorial/controlflow.html\#unpacking-argument-lists}}
\newcommand{\plist}{\url{https://docs.python.org/3/tutorial/datastructures.html\#list-comprehensions}}
\newcommand{\piter}{\url{https://docs.python.org/3/library/itertools.html}}

\title{\textbf{Poda $\alpha$-$\beta$ aplicada em \textit{gomoku}}}
\author{Emmanuel Podestá Jr., Gustavo Zambonin\thanks{
    \texttt{\{emmanuel.podesta,gustavo.zambonin\}@grad.ufsc.br} --- todos os
    algoritmos utilizados podem ser encontrados também
    \href{https://github.com/zambonin/ufsc-ine5430}{neste repositório}.} \\
\small {Inteligência Artificial (UFSC -- INE5430)} \vspace{-5mm}}
\date{}

\begin{document}

\maketitle

\begin{itemize}

    \item \textbf{Funções de heurística e utilidade}

        A função heurística $H(T)$, onde $T$ é um nodo do grafo que
        caracteriza um tabuleiro do jogo \textit{gomoku}, foi definida como
        a soma da quantidade de $n$- uplas, $n \in \{1, 2, 3, 4\}$, afetadas
        por certos fatores multiplicativos.

        Tais fatores derivam-se do que pode ser identificado como `'boas`'
        característica da organização de peças de um determinado jogador:
        alinhamento em algum eixo, falta de obstrução pelo adversário e
        tamanho da $n$-upla.

        A obstrução é codificada como um fator que altera gravemente o valor
        do tabuleiro caso uma $n$-upla não consiga ser estendida pelo jogador
        -- ou seja, se existirem peças do adversário em cada um dos lados
        de modo a impossibilitar essas jogadas. Caso apenas um dos lados
        esteja bloqueado, a punição não é tão severa. Pode ser representada
        como
        \begin{align*}
            s(x) =
            \begin{cases}
                1 \text{ if } x = 2 \\
                \frac{1}{2} \text{ if } x = 1 \\
                \frac{1}{10} \text{ if } x = 0
            \end{cases}
        \end{align*}

        onde $n$ é o número de lados disponíveis para jogada.

        $n$-uplas alinhadas recebem valores de acordo com seu tamanho. Tomando
        $v_n$ como o valor de uma $n$-uplas de tamanho $n$, é possível
        calcular $v_{n + 1}$, assumindo alguns fatos na construção do
        raciocínio:

        \begin{itemize}

            \item $v_{n + 1} > \sum v_n$, para que o algoritmo sempre busque
                construir as maiores $n$-uplas;

            \item um valor inicial não-nulo para $v_1$, a menor $n$-upla que a
                função considera;

            \item existem $15 - n + 1$ possíveis arranjos de elementos
                consecutivos com o mesmo valor em uma palavra binária de
                tamanho $15$;

            \item $\lceil \frac{15^2}{2} \rceil = 113$ é o maior número
                de peças de um jogador num tabuleiro, o que implica em
                $\frac{113}{15} \approx 7.5$ linhas de peças iguais;

            \item existem 3 orientações onde um arranjo pode ser considerado
                como $n$-upla.

        \end{itemize}

        Então, $v_{n+1} = \sum v_n \times 3 \times 7.5 \times 113 \times
        (15 - n + 1)$, e assim tem-se os valores para cada uma das $n$-uplas.
        Codificando este cálculo como uma função $v(t) = v_n$, onde $t$ é uma
        $n$-upla de tamanho $n$ dentro do tabuleiro $T$, tem-se então a função
        heurística final
        \begin{equation*}
        H(T) = \displaystyle\sum_{n=1}^4 v(t) * s(x), \forall t \in T
        \end{equation*}
        onde $x$ é obtido analisando as peças adjacentes à $n$-upla.

        A função de utilidade foi definida de forma semelhante, levando em
        conta a dificuldade de representar todos os nodos possíveis do grafo
        em virtude da quantidade de tabuleiros possíveis. Dessa forma, foi
        necessário especificar uma profundidade limite para o algoritmo
        minimax. Este fato dificulta a introdução de um fator heurístico
        relacionado à profundidade do grafo, pois seria custoso armazenar
        e processar tal informação ao longo da execução do algoritmo.

        Então, tomando $U(T)$ como a função utilidade, estendeu-se a análise
        de $n$-uplas para $n = 5$, aumentando a pontuação final do tabuleiro
        com um grande valor caso tal $n$-upla exista. Assim,
        \begin{equation*}
        U(T) = \displaystyle\sum_{n=1}^5 v(t) * s(x), \forall t \in T
        \end{equation*}
        onde o grande valor obtido quando encontra-se uma $5$-upla apontará
        corretamente a importância do nodo, que neste caso é vitorioso.

    \item \textbf{Otimizações e estratégias}

        O algoritmo minimax com poda $\alpha$-$\beta$ foi utilizado para
        buscar melhores jogadas. A estrutura de dados que representa o
        tabuleiro é uma lista de listas simples, onde cada elemento pode
        ser um inteiro em $0, 1, -1$: respectivamente espaço vazio e dois
        jogadores. Dado uma configuração de entrada, o algoritmo verificará
        possíveis movimentos ao redor das peças já inseridas no tabuleiro,
        armazenará os possíveis movimentos em uma lista, e vários cenários
        de jogadas com cada uma dessas coordenadas serão testados,
        recursivamente. A avaliação de melhor jogada será feita sobre os
        valores de cada um dos tabuleiros. Após o processo de verificação
        e poda, a melhor jogada escolhida, juntamente com sua pontuação,
        é retornada para que o algoritmo possa proceder.

    \item \textbf{Detecção de fim de jogo e sequências}

        A detecção de fim de jogo é feita a partir de um método
        \texttt{victory}, que percorre toda a representação matricial do
        tabuleiro procurando por quíntuplas da mesma cor. Após cada jogada,
        essa verificação é aplicada e $U(T)$ é chamada caso o resultado
        seja verdadeiro. De modo similar, sequências são verificadas com
        uma variante do método acima que observa apenas $n$-uplas de um
        jogador e tamanho, fornecidos como argumentos. Note que este método
        não necessariamente busca por $n$-uplas com peças consecutivas, e
        pode identificar sequências com um espaço vazio como boas jogadas. O
        funcionamento específico tem base em \texttt{itertools.groupby} da
        linguagem Python, que agrupa símbolos semelhantes e sua quantidade em
        um objeto iterável, como uma lista.

    \item \textbf{Decisões de projeto}

        A implementação foi dividida em três classes, cada qual com sua
        função imutável:

        \begin{itemize}

            \item \texttt{game.py} é responsável pela parte de acesso aos
                modos de jogo e validade das entradas, bem como todo o fluxo
                de controle, comunicando-se com o tabuleiro e chamando métodos
                necessários para que jogadores humanos e artificial possam
                progredir no jogo.

            \item \texttt{gomoku\_board.py} é uma implementação simples de um
                tabuleiro para um jogo de \textit{gomoku}, com métodos
                variados para suportar a integração com um algoritmo de busca
                de decisão, utilizando vários recursos da linguagem escolhida,
                tais como \textit{slices}\footnote{\pslice},
				\textit{unpacking}\footnote{\punpack} e
                \textit{list comprehensions}\footnote{\plist}, bem como várias
				funções nativas (em especial, a biblioteca
                \texttt{itertools}\footnote{\piter}).

            \item \texttt{minimax.py} é responsável por construir árvores de
                possibilidades dinamicamente, dado o tabuleiro atual, com
                uma profundidade escolhida previamente, e decidir qual será
                a melhor jogada possível respeitando as funções heurística e
                utilidade. A análise de vitória é feita em todo nodo e não
                apenas nos que são folhas, pois tal situação pode acontecer
                em nodos que são pais; apenas as posições vazias e vizinhas
                às peças de ambos os jogadores são avaliadas, pois deseja-se
                bloquear o jogador adversário e, ao mesmo tempo, as peças do
                jogador atual geralmente estão adjacentes.

        \end{itemize}

    \item \textbf{Limitações}

        \begin{itemize}

            \item As cores representando os jogadores foram fixadas em virtude
                da falta de abstração na identificação de quem fará a
                maximização e minimização; assim, a peça de cor branca sempre
                iniciará, e esta representa o jogador humano.

            \item O desempenho é um problema real, em virtude de todas as
                possibilidades calculadas no algoritmo minimax; seu gargalo
                apresenta-se na função \texttt{neighbor\_board}, pois é
                necessário verificar quais são os espaços válidos ao redor de
                uma dada coordenada a partir do raio desejado e filtrá-los de
                modo que tais espaços possam formar $n$-uplas úteis para o
                jogador.

            \item A profundidade escolhida para o algoritmo minimax também
                implica diretamente em quão rapidamente o jogador artificial
                fará sua jogada, e deve-se escolher um valor ótimo para que o
                algoritmo não demore demais, mas as jogadas ainda sejam
                plausíveis.

            \item A heurística poderia ser refinada para levar em conta a
                profundidade da árvore, porém isso implicaria em uma grande
                mudança na implementação do algoritmo minimax, e portanto a
                sugestão foi descartada.

        \end{itemize}

    \item \textbf{Principais métodos}

        \begin{itemize}

            \item \texttt{minimax.py:ab\_pruning()} implementa o algoritmo
                minimax com poda.

            \item \texttt{gomoku\_board.py:neighbor\_board()} verifica
                vizinhos de uma dada coordenada utilizando
                \texttt{itertools.starmap} para gerar todas as coordenadas
                válidas no formato de "estrela", e filtra posições válidas
                através de uma função anônima.

            \item \texttt{gomoku\_board.py:row\_values()} calcula o valor
                (heurístico) de uma lista de listas de coordenadas; pode ser
                reutilizado para todos os eixos, dado que a construção de tais
                listas sejam corretas.

        \end{itemize}

\end{itemize}

\end{document}
