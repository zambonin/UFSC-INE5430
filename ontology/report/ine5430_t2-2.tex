\documentclass{../../sftex/sftex}

\usepackage{amsmath, graphicx}

\title{Representação de conhecimento e raciocínio}
\author{Emmanuel Podestá Jr., Gustavo Zambonin}
\email{\{emmanuel.podesta,gustavo.zambonin\}@grad.ufsc.br}
\src{https://github.com/zambonin/ufsc-ine5430}
\uniclass{Inteligência Artificial}
\classcode{UFSC-INE5430}

\begin{document}

\maketitle

\begin{itemize}

    \item \textbf{Diferença entre axiomas}

        O axioma \texttt{SubClassOf} determina que uma classe $c_1$ é
        subclasse de uma classe $c_2$. Mais especificamente, $c_1$ é uma
        especialização de $c_2$. Desta forma, $c_2$ irá herdar todas as
        características de $c_1$, contudo $c_1$ não terá, necessariamente, as
        características de $c_2$.

        No domínio deste trabalho, a classe \texttt{Dumplings} é subclasse
        de \texttt{Bait}, herdando suas características e adicionando as
        seguintes: \texttt{(catch only Carp) or (catch only Liberl\_Carp) or
        (catch only Pearlglass)}.

        O axioma \texttt{EquivalentTo} significa que todas as expressões de
        uma classe são equivalentes entre si. Isto possibilita o uso de
        sinônimos entre classes; logo, se $c_1$ for equivalente à $c_2$, ambas
        as classes possuirão as mesmas características.

        No domínio deste trabalho, a classe \texttt{Fisherman} é equivalente
        a \texttt{(can\_use only Rods) and (can\_use max 1 Rods)}.

    \item \textbf{Motor de inferência \emph{HermiT}}

        Para explicar o funcionamento do motor de inferência \emph{Hermit}, é
        necessário apresentar o método dos \emph{tableaux} analíticos. De modo
        a acelerar o processo de verificação de uma fórmula lógica, é possível
        analisar apenas a sua estrutura exterior e aplicar axiomas
        apropriados, que partem da seguinte estratégia: é mais fácil derivar
        contradições para uma fórmula do que provar que todos os seus casos
        são verdadeiros. Assim, tal método sintetiza esta técnica, manipulando
        sintaticamente a fórmula lógica inicial e construindo uma árvore de
        possibilidades, que podem ser contraditórias entre si, assim
        simplificando o resultado final.~\cite{Constable:2012}

        O cálculo \emph{tableau} é um conjunto de regras que ditam como o
        \emph{tableau} pode ser modificado, ou seja, quais axiomas podem ser
        utilizados na construção da árvore de possibilidades. O motor
        \emph{HermiT}~\cite{smh08HermiT}, propõe um novo tipo de cálculo com
        diversas otimizações, de modo a diminuir o ônus causado pela
        complexidade de ontologias, assim conseguindo classificar uma grande
        variedade de modelos.

    \item \textbf{Descrição do domínio}

        O domínio escolhido para este trabalho consiste na modelagem da
        dinâmica de pescaria do jogo \emph{The Legend of Heroes: Trails in the
        Sky Second Chapter}, onde existem varas de pescar que podem utilizar
        apenas alguns tipos de iscas, iscas que só atraem alguns tipos de
        peixes, e peixes que podem ser iscas para outras espécies. Foram
        também adicionadas classes de pescadores de acordo com sua habilidade,
        traduzida pela capacidade de utilizar um conjunto de varas de pescar
        específico. O motor de inferência consegue classificar corretamente
        todas as entidades que podem utilizar uma vara de pescar como humanos
        e pescadores, entidades que ``pescam'' outras como iscas ou peixes,
        e completar uma relação de amizade simétrica.

\end{itemize}

\bibliography{ine5430_t2-2}
\bibliographystyle{plain}

\end{document}
