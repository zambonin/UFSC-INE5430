\documentclass{article}

\usepackage[brazil]{babel}
\usepackage[T1]{fontenc}
\usepackage[a4paper, margin=1.5cm]{geometry}
\usepackage[colorlinks, urlcolor=blue, citecolor=red]{hyperref}
\usepackage[utf8]{inputenc}
\usepackage{enumitem}

\title{\textbf{Representação de conhecimento e raciocínio}}
\author{Emmanuel Podestá Jr., Gustavo Zambonin\thanks{
    \texttt{\{emmanuel.podesta,gustavo.zambonin\}@grad.ufsc.br}} \\
\small {Inteligência Artificial (UFSC -- INE5430)} \vspace{-5mm}}
\date{}

\begin{document}

\maketitle

\begin{enumerate}[label=\textbf{\arabic*})]

    \item Redes semânticas são úteis para representar naturalmente relações
        entre entidades; contudo, com um grande aumento nas tarefas a serem
        resolvidas, tais relações podem tornar-se muito complexas. Com o
        surgimento do conceito de orientação à objetos, muitas linguagens
        começaram a adotar esse conceito. A área de inteligência artificial
        também começou a utilizar deste paradigma, já que resolvia um de seus
        maiores problemas existentes um método adequado para representar o
        conhecimento, isto é, capturar informações sobre objetos. Note que
        essa característica é inerente da programação orientada à objetos.
        Desta forma, Minsky \cite{1975:MFS:980190.980222} desenvolveu uma
        base para IA seguindo essa ideia. Assim, as relações entre as
        entidades e os nodos começaram a parecer com frames para manter
        conhecimento sobre tudo no sistema.

    \item

        \begin{enumerate}

            \item De acordo com \cite[p. 23, 336]{Russell:2009:AIM:1671238},
                sistemas de produção podem ser caracterizados como programas
                que consomem conjuntos de regras sobre comportamento, na forma
                de cláusulas em lógica de primeira ordem, com asserções
                similares a construtos se-então ($\Rightarrow$), e produzem
                inferências de acordo com este conhecimento. Similarmente,
                sistemas especialistas também inferem de acordo com regras da
                forma se-então; geralmente, um sistema especialista emprega um
                motor de inferência na forma de um sistema de produção, para
                processar as regras da base de conhecimento.

            \item Sistemas especialistas podem ser utilizados em diversas
                facetas da sociedade; exemplos encontrados após uma rápida
                revisão bibliográfica (Google Scholar) incluem sistemas na
                área de ciências biológicas (identificação de padrões em DNA,
                e similaridades entre plantas); aprendizagem estudantil
                (interação com agentes digitais para facilitar tal processo);
                automação industrial (limite de temperatura para fornalhas em
                uma metalúrgica); bolsas de valores (denotar valores para
                investimentos e outras movimentações financeiras) etc.

        \end{enumerate}

    \item Uma melhor associação seria com IA forte, pois nessa representação
        tem-se descrições simbólicas e manipulação destas; isto é, algo mais
        conceitual e abstrato. O foco sobre o resultado final e a simulação
        intermediária do conhecimento é diminuído, dando espaço para a
        representação da informação sobre o domínio.

\end{enumerate}

\bibliography{ine5430_t2}
\bibliographystyle{plain}

\end{document}
